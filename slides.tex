\documentclass{beamer}
%\documentclass[handout,t]{beamer}

\batchmode
% \usepackage{pgfpages}
% \pgfpagesuselayout{4 on 1}[letterpaper,landscape,border shrink=5mm]

\usepackage{ctex,amsmath,amssymb,enumerate,epsfig,bbm,calc,color,ifthen,capt-of}
\usepackage{url} 
\usepackage{hyperref} 
\usepackage{geometry}

\usetheme{Berlin}
%\usecolortheme{mit}

\title{媒体云转码的演进:\\\Small{MapReduce、DASH与稳定婚姻}}
\author{Alan Zhuang\\
\href{mailto:cheedoong@acm.org}{\nolinkurl{cheedoong@acm.org}}\\
}
\date{\today}
\pgfdeclareimage[height=0.25cm]{mit-logo}{tencent_alpha.png}
\logo{\pgfuseimage{mit-logo}\hspace*{0.1cm}}

%\setlist[itemize]{noitemsep}

\AtBeginSection[]
{
\begin{frame}<beamer>
\frametitle{Outline}
\tableofcontents[currentsection]
\end{frame}
}
\beamerdefaultoverlayspecification{<+->}
% -----------------------------------------------------------------------------
\begin{document}
% -----------------------------------------------------------------------------

\frame{\titlepage}

\section[Outline]{}
\begin{frame}{Outline}
\tableofcontents
\end{frame}

% -----------------------------------------------------------------------------
\section{背景}
\subsection{}
\begin{frame}{互动}
\begin{itemize}
\item Q: 谁拥有以下设备之一?
	\begin{itemize}
	\item 小米3联通版、红米Note
	\item 华为荣耀X1、荣耀3X
	\item 三星Galaxy S4
	\item Google/LG Nexus 5
	\item LG G2
	\item 酷派大神F1/8297、9976A
	\end{itemize}
\item  举过手的,恭喜大家!\\
原因请继续听...
\end{itemize}
\end{frame}
\begin{frame}{多屏时代的挑战}
\begin{itemize}
%\pause
\item 多种平台\\ %\pause
\includegraphics[height=1.6cm]{fig/PCs.png}\pause
\includegraphics[height=1.0cm]{fig/mobile-bc.png}\pause
\includegraphics[height=1.1cm]{fig/streaming-bc.png}\pause
\includegraphics[height=1.6cm]{fig/video_quality-bc.png} \pause
\item 多种屏幕大小\\ %\pause
\includegraphics[height=1.2cm]{fig/screen_sizes.jpg}\pause
\includegraphics[height=2.2cm]{fig/480_to_4KVideo.jpg}\pause
\includegraphics[height=2.2cm]{fig/4k_video.jpg} \pause
\end{itemize}
\end{frame}
\begin{frame}{多屏时代的挑战} 
\begin{itemize}
\item 多种码率\\ %\pause
\includegraphics[height=1.9cm]{fig/bitrate_tencent.png}\hspace*{0.4cm}\pause
\includegraphics[height=1.9cm]{fig/bitrate_youku.png}\hspace*{0.4cm}\pause
\includegraphics[height=1.9cm]{fig/bitrate_sohu.png}\hspace*{0.4cm}\pause
\includegraphics[height=1.9cm]{fig/bitrate_qiyi.png}\pause   
\item 多种解码能力\\ %\pause
%!!!!!!m目前支持HEVC的手机:Galaxy S4, LG G2, Nexus5, 小米3联通版, 红米1S/2... 
%\begin{table}
{\scriptsize
\begin{center}
\begin{tabular}{l|llll} %\toprule
\hline
联发科芯片& MT6572 & MT6582 & MT6588 & MT6592 \\ %\midrule
\hline
Display  & 960$\times$540P & 1280$\times$720P & 1920$\times$1280P & 1920$\times$1280P \\
H.264 Decode   & 720P@30fps & 1080P@30fps  & 1080P@30fps  & 1080P@30fps \\ 
HEVC Decode   &  N/A &  N/A  & 720P@30fps  & 720P@30fps \\ 
\hline
%\bottomrule
\end{tabular}
}
\end{center}
%\end{table}
\end{itemize}
\end{frame}
\begin{frame}{多屏时代的挑战} 
\begin{itemize}
\item 不同封装容器支持\\ %\pause
mp4, mkv, avi, flv, wmv, rmvb, webm, mpeg-ts... \pause
\item 不同编码标准支持\\ \pause
H.264(AVC), H.265(HEVC), VC-1, AVS, VP8/9, RealVideo...
\pause
\includegraphics[height=3.5cm]{fig/encoding_standards.png}\pause
\end{itemize}
\end{frame}
\begin{frame}{多屏时代的挑战} 
\begin{itemize}
\item 巨头角力\\ %\pause
\includegraphics[height=5cm]{fig/competition.jpg}\pause
\end{itemize}
\end{frame}
\begin{frame}{幸运的是,}
\pause
绝大多数设备都支持:
\pause
\begin{itemize}
\item 编码标准\\ \pause
H.264/AVC (ISO/IEC 14496-10;  ITU-T H.264; MPEG-4 Part 10).
\item 封装容器\\ \pause
MP4 (ISO/IEC 14496-14; MPEG-4 Part 14).
\end{itemize}
\pause
所以,我们需要把用户/编辑上传的各种不同封装、不同编码的,一般码率比较高的源视频转成若干种适合不同设备的不同码率的H.264编码、MP4封装的视频。
\end{frame}
\begin{frame}{但是,}
\pause
\begin{itemize}
\item 媒体转码是件极其消耗计算资源的工作\\ %\pause
	尤其视频编码,对于目前常见的支持SSE4指令集的x86/x64 CPU的机器:
	\begin{itemize}
	\item  编码H.264视频需要耗费播放时长的1/3到2/3
	\item  编码H.265视频需要耗费播放时长的30+倍
	\item  单个CPU核通常最多可跑1--2个编码任务
	\end{itemize}
\item 媒体文件大,再加上多码率副本,极其消耗存储
\item 潜在的带宽消耗
\end{itemize}
\end{frame}

\begin{frame}{以往的解决:并行与分布式}
\pause
Criteria: \pause
\begin{itemize}
\item 单机内\\
	\begin{itemize}
	\item 功能划分、数据局部性\\
	宏块组粒度的并行
	\item 内存访问、CPU核心/Cache拓扑结构和转码格式\\
	帧级或GOP(图像组)级并行\\
	对NUMA机器特别友好
	\end{itemize}
\item 分布式转码
	\begin{itemize}
		\item 存储
		\item 路由
		\item 资源分配、任务调度
	\end{itemize}
\end{itemize}
\end{frame}

% -----------------------------------------------------------------------------
\section{从Cloud Transcoder到TranscX}
\subsection{前腾讯研究院Cloud Transcoder}
\begin{frame}{前腾讯研究院Cloud Transcoder}
\textbf{Done 2011-. Gale Huang et al. Cloud transcoder: bridging the format and resolution gap between internet videos and mobile devices. ACM NOSSDAV 2012.}\\\pause
\begin{center}
\includegraphics[height=2.8cm]{fig/clouder-transcoder_principle.png}\\\pause
\includegraphics[scale=0.36]{fig/cloud_transcoder_nossdav_affiliation.png}
\end{center}
\end{frame}

\begin{frame}
\begin{center}
\includegraphics[width=11.5cm]{fig/cloud-transcoder_arch.png}
\end{center}
\end{frame}
\begin{frame}
\begin{center}
\includegraphics[width=11.5cm]{fig/cloud-transcoder_arch_details.png}
\end{center}
\end{frame}
\begin{frame}
\begin{center}
\includegraphics[width=10cm]{fig/cloud-cache_hardware.png}
\end{center}
\end{frame}
\begin{frame}{评价}
\begin{itemize}
	\item 整体上是个设计优秀的系统
	\item 但还存在一些问题:
	\begin{itemize}
		\item 下载、转码、任务分发、任务管理都需要专门的机器资源
		\item 下载后,数据还需要再次传输到相应的转码机器
		\item 切片的开销
		\item 划分的各模块,增加了部署复杂性,限制了可扩展性
		\item 存储部分依赖NFS,不能支持海量的媒体资源
		\item 转码完成后,分发是个问题
	\end{itemize}
\end{itemize}
\end{frame}

\subsection{架平流媒体TranscX}
\begin{frame}{架平流媒体TranscX}
TranscX: 
\begin{itemize}
	\item Transcoding eXpress/eXperience/eXtreme...; transc(x)
\end{itemize}
\begin{center}
\includegraphics[scale=0.20]{fig/GOP.pdf}
\end{center}
\begin{itemize}
\item GOP级的并行,不需要真正“切割”\\
Media head parsing, <begin\_time, end\_time> or GOP\_id $\rightarrow$ <start\_offset, size>
\end{itemize}
\end{frame}

\begin{frame}
\begin{itemize}
\item I\\
\includegraphics[scale=0.40]{fig/cjk1.jpg}
\end{itemize}
\end{frame}
\begin{frame}
\begin{itemize}
\item B\\
\includegraphics[scale=0.40]{fig/cjk2.jpg}
\end{itemize}
\end{frame}
\begin{frame}
\begin{itemize}
\item B\\
\includegraphics[scale=0.40]{fig/cjk3.jpg}
\end{itemize}
\end{frame}
\begin{frame}
\begin{itemize}
\item P\\
\includegraphics[scale=0.40]{fig/cjk4.jpg}
\end{itemize}
\end{frame}
\begin{frame}
\begin{itemize}
\item I: IDR\\
\includegraphics[scale=0.40]{fig/bfj1.jpg}
\end{itemize}
\end{frame}
\begin{frame}
\begin{itemize}
\item B\\
\includegraphics[scale=0.40]{fig/bfj2.jpg}
\end{itemize}
\end{frame}
\begin{frame}
\begin{itemize}
\item B...\\
\includegraphics[scale=0.40]{fig/bfj3.jpg}
\end{itemize}
\end{frame}

\begin{frame}
\begin{itemize}
\item Job/Map/Thread三级精确控制,CPU \& I/O limitation\\
避免影响存储集群的现网服务
\item 对DASH和实时在线直播的支持
\item 计算向存储迁移:云计算的理想状态\\
而且转码工作在存储机器上完成,避免了大部分的数据传输/拷贝,效果接近本地计算
\item 后来逐步支持了WeChat、Qzone和Weishi
\item 用MapReduce框架统一了起来,减少了调度的复杂性,增强了可扩展性\\
通过TFS/CFS分布式文件系统和网络协议,可处理PB级别的媒体资源
\item 复用存储服务器的空间计算资源,减少碳排放(绿色计算)
\end{itemize}
\end{frame}

\begin{frame}
\begin{center}
\includegraphics[height=6.4cm]{fig/TranscX.pdf}
\end{center}
\end{frame}
\begin{frame}
\begin{center}
\includegraphics[height=6.2cm]{fig/TranscX_detail.pdf}
\end{center}
\end{frame}

\section{DASH与稳定婚姻}
\subsection{DASH}
\begin{frame}{Why DASH?}
DASH: Dynamic Adaptive Streaming over HTTP.
\pause
\begin{itemize}
\item 庞大的文件头,导致在线播放时较大的initial/VCR delay
\includegraphics[height=4cm]{fig/MP4_boxes_detail.jpg}
\includegraphics[height=4cm]{fig/fragmented_mp4.jpg}
\end{itemize}
\end{frame}
\begin{frame}
\begin{itemize}
\item 非平滑的码率间切换 due to varying download speed \\
\includegraphics[height=5.4cm]{fig/download_speed.png}
\end{itemize}
\end{frame}
\begin{frame}{DASH}
DASH: Dynamic Adaptive Streaming over HTTP. \\
几种DASH标准:
\pause
\begin{itemize}
\item Apple HLS (HTTP Live Streaming) 2009
\item Microsoft HSS (HTTP Smooth Streaming) 2010
\item Adobe HDS (HTTP Dynamic Streaming) 2010
\item MPEG-DASH (ISO/IEC 23009-1) 2012
\end{itemize}
\end{frame}
\begin{frame}{Apple HLS}
\begin{itemize}
\item Architecture
\begin{center}
\includegraphics[height=4.5cm]{fig/hls_arch.jpg}
\end{center}
\end{itemize}
\end{frame}
\begin{frame}{Apple HLS}
\begin{itemize}
\item Segment Indexing
\begin{center}
\includegraphics[height=4.5cm]{fig/hls_indexing.jpg}
\end{center}
\end{itemize}
\end{frame}

\begin{frame}{MPEG-DASH}
\begin{itemize}
\item Architecture
\includegraphics[height=4.5cm]{fig/MPEG-DASH_arch.png}
\end{itemize}
\end{frame}
\begin{frame}{MPEG-DASH}
\begin{itemize}
\item Data Model
\includegraphics[height=4.5cm]{fig/mpeg-dash_data_model.png}
\end{itemize}
\end{frame}
\begin{frame}{MPEG-DASH}
\begin{itemize}
\item Segment Indexing
\includegraphics[height=5cm]{fig/mpeg-dash_indexing.png}
\end{itemize}
\end{frame}

\begin{frame}{架平流媒体对DASH的使用}
\begin{itemize}
\item 尽量只转封装,不转编码
\item 实时、按需转封装/转码
\item 使用/自写的多种协议,解决并发I/O流少的问题
	\begin{description}
	\item[tmpfs] 最简单、方便,对Cache管理要求不强的场合
	\item[in\_mem] in memory, 进程内的编解码I/O
	\item[shm\_mem] shared memory, 多进程的编解码I/O,可自定义Cache管理、淘汰算法
	\item[AVIOContext] FFMPEG built-in
	\item[async\_http] 适用直播,数据到达不可预期
\end{description}
\item FPGA HEVC Encoding (proposed)
\end{itemize}
\end{frame}

\subsection{稳定匹配}
\begin{frame}{Motivation}
DASH和Wechat, Weishi, Qzone等社交UGC视频转码中存在的问题:
\begin{itemize}
\item 视频长度较短\\
一般仅一个到若干个GOP,MapReduce时仅需要一个Mapper
\item 但MapReduce任务调度的延时较大\\
有时甚至超过播放时长。可通过Linux Container, Docker守护进程的实时MapReduce来缓解,但非根本解决之道
\item 大部分视频的未来访问少于X次\\
还有比较全码率转码+全量分发么?
\end{itemize}
\end{frame}
\begin{frame}{大数据分析,得出结论:}
\begin{itemize}
	\item 对数千台后台服务器连续一周运行状态的测量:\\
		\begin{itemize}
		\item 不仅DC,OC的多数服务器大部分时间CPU负载也较低\\
		甚至包括某些HTTP服务器
		\item 多数服务器的CPU负载基本不会出现突变\\
		CPU负载随时间相对平稳,但不同地区的平均负载存在差异
		\end{itemize}
	\item 对应时间所有用户访问和调度情况的测量:\\
		\begin{itemize}
		\item 用户对CDN的region选择存在偏好\\
		由于网络拓扑关系,用户去不同的CDN边缘节点下载的速率各不相同,而目前的调度算法,至少能得到一个次优解
		\item 不同不同地区的用户对码率存在偏好\\
		在前面次优解的前提下,不同地区得到平均服务质量有差异
	\end{itemize}
\end{itemize}
\end{frame}
\begin{frame}{带宽和CPU使用率示例}
\includegraphics[height=2.4cm]{fig/bandwidth.png}\\
\includegraphics[height=3.2cm]{fig/cpu_usage.png}
\end{frame}
\begin{frame}{启发}
\begin{itemize}
\item CDN的大部分服务器是否也可用来做转码?\\
\item 是否可以按需转码,按需分发?\\
on-line, on-the-fly, on-demand?
\item 当前CDN节点实在是没有所需码率片段时...\\
是否可以给一个不高于所需码率的最高码率副本来替代?
\end{itemize}
\end{frame}

\begin{frame}{优化问题}
\begin{itemize}
\item 用户重定向\\
用户当前请求应该被调度到哪个CDN节点来服务?资源应尽量在哪儿转码?服务器负载、带宽利用、用户体验,哪个更重要?
\item 哪些资源的哪些片段最需要被转码\\
用户最需要的,能尽量让用户得到最大码率的,当前副本最少的?消耗服务器计算资源少的(节能/碳环保)?
\item 转码后的片段应当怎样在CDN云中分发\\
怎样定义分发代价?怎样让代价最小?怎样尽可能不超出购买带宽量?
\end{itemize}
\end{frame}
\begin{frame}{我们的启发式算法}
\begin{itemize}
\item 启发式规则和稳定匹配(婚姻)结合
	\begin{itemize}
		\item 启发式:用贪心规则给出NP难问题的近似最优解
		\item 稳定匹配:扩展的稳定婚姻问题\\
			\begin{itemize}
					\item 女追男---尽量利于女神\\
						女:用户;男:CDN边缘节点
					\item 男方具有多容量
					\item 稳定性与帕累托效率
			\end{itemize}
	\end{itemize}
\end{itemize}
\end{frame}
\begin{frame}{算法应用}
准备与预测框架\\
\begin{itemize}
\item 用户对所有CDN边缘节点的偏好\\
根据过去的测速数据来Rank用户对所有CDN边缘节点的偏好
\item 每段视频未来一段时间的访问频率\\
根据CDN到用户的带宽和过去一段时间用户的访问频次预测未来一段时间所有有可能有访问的视频的访问频次
\item 未来一小段时间服务器的期望空间计算资源\\
根据分工作日/周末两种自回归模型预测未来一段时间每个OC节点服务器的期望CPU负载
\end{itemize}
\end{frame}
\begin{frame}{整体架构与“女追男”模型}
\includegraphics[height=5.6cm]{fig/transcoding_delivery.pdf}
\end{frame}
\begin{frame}{至关重要:Satisfying Women}
\includegraphics[height=5.6cm]{fig/massage2.png}
\end{frame}
\begin{frame}{效果}
\begin{itemize}
\item 44.8\%的用户会享受更高码率的版本\\
	相比传统服务器负载均衡的算法
\item 4.5倍的用户享受到最高的可能码率\\
	相比传统服务器负载均衡的算法
\item 减少42.2\%接受和当前带宽不匹配码率的可能\\
	相比传统FIFO(先来先服务)的算法
\item 减少约80\%的转码计算资源消耗\\
	对于微视、腾讯视频这种4码率副本的场景。码率副本数越高,效果越明显。(Qzone两码率副本场合,减少约40\%)
\item 副本传输的带宽不随分段数显著增长\\
	事实上,当分段数$n$足够大后,随$n$几乎不变了。
\end{itemize}
\end{frame}
\begin{frame}{Publications}
\begin{center}
\includegraphics[width=10cm]{fig/infocom.jpg}\\\pause
...Alan Zhuang... Joint Online Transcoding and Geo-distributed Delivery for Dynamic Adaptive Streaming. IEEE Transactions on Parallel and Distributed Systems (TPDS). 2014. (will appear)\\\pause
\includegraphics[width=10cm]{fig/tpds_25yr.jpg}\\\pause
\end{center}
\end{frame}

\section{Future Vision}
\subsection{和一些新技术的结合}
\begin{frame}{和一些新技术的结合}
\begin{itemize}
\item WebRTC\\
Client-end caching \& delivery \& broadcasting
\item HTML5 WebSocket, HTTP 2.0\\
会话保持,多流传输
\item Social Could TV\\
\item SVC, MVC, HEVC, NC\\
编码、多副本多版本机制、多视角、传输
\end{itemize}
\end{frame}
\subsection{Acknowledgment}
\begin{frame}{Acknowledgment}
本slides中部分内容源自与以下人员的协作/交流:
\begin{itemize}
\item 架平流媒体\\
Leon Ouyang, Devin Zeng, Guita Zhang, Kernel He, Chris Su
\item Tsinghua-Tencent Joint Lab.\\ 
Zhi Wang
\item SNG社交平台部\\
Stone Huang
\item NTU\\
Young Wen
\end{itemize}
\end{frame}
% -----------------------------------------------------------------------------
\end{document}
