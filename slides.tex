\documentclass{beamer}
%\documentclass[handout,t]{beamer}

\batchmode
% \usepackage{pgfpages}
% \pgfpagesuselayout{4 on 1}[letterpaper,landscape,border shrink=5mm]

\usepackage{ctex,amsmath,amssymb,enumerate,epsfig,bbm,calc,color,ifthen,capt-of}
\usepackage{url} 
\usepackage{hyperref} 
\usepackage{geometry}

\usetheme{Berlin}
%\usecolortheme{mit}

\title{媒体云转码的演进:\\\Small{MapReduce、DASH与稳定婚姻}}
\author{Alan Zhuang\\
\href{mailto:cheedoong@acm.org}{\nolinkurl{cheedoong@acm.org}}\\
}
\date{\today}
\pgfdeclareimage[height=0.25cm]{mit-logo}{tencent_alpha.png}
\logo{\pgfuseimage{mit-logo}\hspace*{0.1cm}}

%\setlist[itemize]{noitemsep}

\AtBeginSection[]
{
\begin{frame}<beamer>
\frametitle{Outline}
\tableofcontents[currentsection]
\end{frame}
}
\beamerdefaultoverlayspecification{<+->}
% -----------------------------------------------------------------------------
\begin{document}
% -----------------------------------------------------------------------------

\frame{\titlepage}

\section[Outline]{}
\begin{frame}{Outline}
\tableofcontents
\end{frame}

% -----------------------------------------------------------------------------
\section{背景}
\subsection{}
\begin{frame}{多屏时代的挑战}
\begin{itemize}
%\pause
\item 多种平台\\ %\pause
\includegraphics[height=1.6cm]{fig/PCs.png}\pause
\includegraphics[height=1.0cm]{fig/mobile-bc.png}\pause
\includegraphics[height=1.1cm]{fig/streaming-bc.png}\pause
\includegraphics[height=1.6cm]{fig/video_quality-bc.png} \pause
\item 多种屏幕大小\\ %\pause
\includegraphics[height=1.2cm]{fig/screen_sizes.jpg}\pause
\includegraphics[height=2.2cm]{fig/480_to_4KVideo.jpg}\pause
\includegraphics[height=2.2cm]{fig/4k_video.jpg} \pause
\end{itemize}
\end{frame}
\begin{frame}{多屏时代的挑战} 
\begin{itemize}
\item 多种码率\\ %\pause
\includegraphics[height=1.9cm]{fig/bitrate_tencent.png}\hspace*{0.4cm}\pause
\includegraphics[height=1.9cm]{fig/bitrate_youku.png}\hspace*{0.4cm}\pause
\includegraphics[height=1.9cm]{fig/bitrate_sohu.png}\hspace*{0.4cm}\pause
\includegraphics[height=1.9cm]{fig/bitrate_qiyi.png}\pause   
\item 多种解码能力\\ %\pause
%\begin{table}
{\scriptsize
\begin{center}
\begin{tabular}{l|llll} %\toprule
\hline
联发科芯片& MT6572 & MT6582 & MT6588 & MT6592 \\ %\midrule
\hline
Display  & 960$\times$540P & 1280$\times$720P & 1920$\times$1280P & 1920$\times$1280P \\
H.264 Decode   & 720P@30fps & 1080P@30fps  & 1080P@30fps  & 1080P@30fps \\ 
HEVC Decode   &  N/A &  N/A  & 720P@30fps  & 720P@30fps \\ 
\hline
%\bottomrule
\end{tabular}
}
\end{center}
%\end{table}
\end{itemize}
\end{frame}
\begin{frame}{多屏时代的挑战} 
\begin{itemize}
\item 不同封装容器支持\\ %\pause
mp4, mkv, avi, flv, wmv, rmvb, webm, mpeg-ts... \pause
\item 不同编码标准支持\\ \pause
H.264(AVC), H.265(HEVC), VC-1, AVS, VP8/9, RealVideo...
\pause
\includegraphics[height=3.5cm]{fig/encoding_standards.png}\pause
\end{itemize}
\end{frame}
\begin{frame}{多屏时代的挑战} 
\begin{itemize}
\item 巨头角力\\ %\pause
\includegraphics[height=5cm]{fig/competition.jpg}\pause
\end{itemize}
\end{frame}
\begin{frame}{幸运的是,}
\pause
决大多数设备都支持:
\pause
\begin{itemize}
\item 编码标准\\ \pause
H.264/AVC (ISO/IEC 14496-10;  ITU-T H.264; MPEG-4 Part 10).
\item 封装容器\\ \pause
MP4 (ISO/IEC 14496-14; MPEG-4 Part 14).
\end{itemize}
\pause
所以,我们需要把用户/编辑上传的各种不同封装、不同编码的,一般码率比较高的源视频转成若干种适合不同设备的不同码率的H.264编码、MP4封装的视频。
\end{frame}
\begin{frame}{但是,}
\pause
\begin{itemize}
\item 媒体转码是件及其消耗计算资源的工作\\ %\pause
	尤其视频编码,对于目前常见的支持SSE4指令集的x86/x64 CPU的机器:
	\begin{itemize}
	\item  编码H.264视频需要耗费播放时长的1/3到2/3
	\item  编码H.265视频需要耗费播放时长的30+倍
	\item  单个CPU核通常最多可跑1--2个编码任务
	\end{itemize}
\item 媒体文件大,再加上多码率副本,极其消耗存储
\item 潜在的带宽消耗
\end{itemize}
\end{frame}

\begin{frame}{解决}
\pause
Criteria: \pause
\begin{itemize}
\item 单机内\\
	\begin{itemize}
	\item 功能划分、数据局部性\\
	宏块组粒度的并行
	\item 内存访问、CPU核心/Cache拓扑结构和转码格式\\
	帧级或GOP(图像组)级并行\\
	对NUMA机器特别友好
	\end{itemize}
\item 分布式转码
	\begin{itemize}
		\item 存储
		\item 路由
		\item 任务调度
	\end{itemize}
\end{itemize}
\end{frame}

% -----------------------------------------------------------------------------
\section{从Cloud Transcoder到TranscX}
\subsection{前腾讯研究院Cloud Transcoder}
\begin{frame}{前腾讯研究院Cloud Transcoder}
\textbf{Done 2011-. Gale Huang et al. Cloud transcoder: bridging the format and resolution gap between internet videos and mobile devices. ACM NOSSDAV 2012.}\\\pause
\begin{center}
\includegraphics[height=2.8cm]{fig/clouder-transcoder_principle.png}\\\pause
\includegraphics[scale=0.36]{fig/cloud_transcoder_nossdav_affiliation.png}
\end{center}
\end{frame}

\begin{frame}
\begin{center}
\includegraphics[width=11.5cm]{fig/cloud-transcoder_arch.png}
\end{center}
\end{frame}
\begin{frame}
\begin{center}
\includegraphics[width=11.5cm]{fig/cloud-transcoder_arch_details.png}
\end{center}
\end{frame}
\begin{frame}
\begin{center}
\includegraphics[width=10cm]{fig/cloud-cache_hardware.png}
\end{center}
\end{frame}
\subsection{架平流媒体TranscX}
\begin{frame}{架平流媒体TranscX}
TranscX: 
\begin{itemize}
	\item Transcoding eXpress/eXperience/eXtreme... 
	\item transc(x)
\end{itemize}

\begin{center}
\includegraphics[scale=0.22]{fig/GOP.pdf}
\end{center}
\begin{itemize}
\item GOP-level parallelism without REAL splitting
\end{itemize}
\end{frame}
\begin{frame}
\begin{itemize}
\item Job/Map/Thread, accurate control and CPU \& I/O limitation
\item Real-time support for DASH and Live Broadcasting
\item Migration from Computing to Storage: soul of cloud tech
\item It later supported WeChat, Qzone and Weishi
\item 用MapReduce框架统一了起来,减少了调度的复杂性,增强了可扩展性
\item 复用了存储服务器的空间计算资源,节省了成本
\end{itemize}
\end{frame}

\begin{frame}
\includegraphics[scale=0.33]{fig/TranscX.pdf}\hspace*{0.1cm}
\includegraphics[scale=0.33]{fig/TranscX_detail.pdf}
\end{frame}

\section{DASH与稳定婚姻}
\subsection{DASH}
\begin{frame}{Why DASH?}
DASH: Dynamic Adaptive Streaming over HTTP.
\pause
\begin{itemize}
\item 庞大的文件头,导致在线播放时较大的initial/VCR delay
\includegraphics[height=4cm]{fig/MP4_boxes_detail.jpg}
\includegraphics[height=4cm]{fig/fragmented_mp4.jpg}
\end{itemize}
\end{frame}
\begin{frame}
\begin{itemize}
\item 非平滑的码率间切换 due to varying download speed
\includegraphics[height=4.5cm]{fig/download_speed.png}
\end{itemize}
\end{frame}
\begin{frame}{DASH}
DASH: Dynamic Adaptive Streaming over HTTP. \\
几种DASH标准:
\pause
\begin{itemize}
\item Apple HLS (HTTP Live Streaming) 2009
\item Microsoft HSS (HTTP Smooth Streaming) 2010
\item Adobe HDS (HTTP Dynamic Streaming) 2010
\item MPEG-DASH (ISO/IEC 23009-1) 2012
\end{itemize}
\end{frame}
\begin{frame}{Apple HLS}
\begin{itemize}
\item Architecture
\begin{center}
\includegraphics[height=4.5cm]{fig/hls_arch.jpg}
\end{center}
\end{itemize}
\end{frame}
\begin{frame}{Apple HLS}
\begin{itemize}
\item Segment Indexing
\begin{center}
\includegraphics[height=4.5cm]{fig/hls_indexing.jpg}
\end{center}
\end{itemize}
\end{frame}

\begin{frame}{MPEG-DASH}
\begin{itemize}
\item Architecture
\includegraphics[height=4.5cm]{fig/MPEG-DASH_arch.png}
\end{itemize}
\end{frame}
\begin{frame}{MPEG-DASH}
\begin{itemize}
\item Data Model
\includegraphics[height=4.5cm]{fig/mpeg-dash_data_model.png}
\end{itemize}
\end{frame}
\begin{frame}{MPEG-DASH}
\begin{itemize}
\item Segment Indexing
\includegraphics[height=5cm]{fig/mpeg-dash_indexing.png}
\end{itemize}
\end{frame}

\begin{frame}{架平流媒体对DASH的使用}
\begin{itemize}
\item 尽量只转封装,不转编码
\item 实时、按需转封装/转码码
\end{itemize}
\end{frame}

\subsection{稳定匹配}
\begin{frame}{Motivation}
对大数据的分析,得出结论:
\begin{itemize}
\item 用户对CDN的region选择存在偏好\\
由于网络拓扑关系,用户去不同的CDN边缘节点下载的速率各不相同,而目前的调度算法,至少能得到一个次优解
\item 不同不同地区的用户对码率存在偏好\\
在前面次优解的前提下,不同地区得到的平均服务质量有显著差异
\end{itemize}
\end{frame}

\begin{frame}{优化问题}
\begin{itemize}
\item 选择选择...\\
to be written...
\item 稳定稳定匹配\\
to be written...
\end{itemize}
\end{frame}
% -----------------------------------------------------------------------------
\end{document}
